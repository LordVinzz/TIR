% This is samplepaper.tex, a sample chapter demonstrating the
% LLNCS macro package for Springer Computer Science proceedings;
% Version 2.21 of 2022/01/12
%
\documentclass[runningheads]{llncs}
%
\usepackage[T1]{fontenc}
% T1 fonts will be used to generate the final print and online PDFs,
% so please use T1 fonts in your manuscript whenever possible.
% Other font encondings may result in incorrect characters.
%
\usepackage{graphicx}
% Used for displaying a sample figure. If possible, figure files should
% be included in EPS format.
%
% If you use the hyperref package, please uncomment the following two lines
% to display URLs in blue roman font according to Springer's eBook style:
%\usepackage{color}
%\renewcommand\UrlFont{\color{blue}\rmfamily}
%\urlstyle{rm}
%
\begin{document}
%
\title{Contribution Title}
%
%\titlerunning{Abbreviated paper title}
% If the paper title is too long for the running head, you can set
% an abbreviated paper title here
%
\author{First Author\inst{1}\orcidID{0000-1111-2222-3333} \and
Second Author\inst{2,3}\orcidID{1111-2222-3333-4444} \and
Third Author\inst{3}\orcidID{2222--3333-4444-5555}}
%
\authorrunning{F. Author et al.}
% First names are abbreviated in the running head.
% If there are more than two authors, 'et al.' is used.
%
\institute{Université Toulouse III - Paul Sabatier, 118 route de Narbonne, 31062 TOULOUSE CEDEX 9
\url{https://www.univ-tlse3.fr/}}
%
\maketitle              % typeset the header of the contribution
%
\begin{abstract}
The abstract should briefly summarize the contents of the paper in
150--250 words.

\keywords{First keyword  \and Second keyword \and Another keyword.}
\end{abstract}
%
%
%
\section{First Section}
\subsection{A Subsection Sample}

%
% ---- Bibliography ----
%
% BibTeX users should specify bibliography style 'splncs04'.
% References will then be sorted and formatted in the correct style.
%
% \bibliographystyle{splncs04}
% \bibliography{mybibliography}
%
\begin{thebibliography}{8}
\bibitem{}
Nguyen, C., Scherzer, D., Ritschel, T. and Seidel, H. : Material Editing in Complex Scenes by Surface Light Field Manipulation and Reflec-
tance Optimization

\bibitem{}
Schmidt, T., Novák, J., Meng, J., Kaplanyan., Reiner, T., Nowrouzezahrai, D. and Dachsbacher, C.: Path-space manipulation of physically-based light transport

\bibitem{}
Subileau, T., Vanderhaeghe, D., Mellado, N. and Paulin, M. : RayPortals : a light transport editing framework 5

\bibitem{}
Deschaintre, V., Rushmeier, H., Guerrero, P., Hašan, M. and Hu, Y.: Controlling Material Appearance by Examples 6

\bibitem{}
Desrichard, F.: Analysis of the path space for light and shadow compositing 7

\bibitem{}
Lagunas, M. and Subias, J.: In-the-wild Material Appearance Editing using Perceptual Attributes 8

\bibitem{}
Schmidt, T., Pellacini, F., Nowrouzezahrai, D. : State of the Art in Artistic Editing of Appearance, Lighting, and Material 9

\bibitem{}
Dorsey, J., Sillion, F., Greenberg, D.: Design and simulation of opera lighting and projection effects

\bibitem{}
Obert, J., Křrivánek, J., Pellacini, F., Sýkora, D. and Pattanaik, S.: iCheat: A Representation for Artistic Control of Indirect Cinematic Lighting

\bibitem{}
Lagunas, M. and Subias, J.: In-the-wild Material Appearance Editing using Perceptual Attributes

\end{thebibliography}
\end{document}
